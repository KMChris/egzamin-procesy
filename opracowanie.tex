\documentclass[10pt,twocolumn]{mwart}
%\documentclass[11pt,twocolumn]{article}
\usepackage[margin=1.7cm]{geometry}
\usepackage{polski}
\usepackage[utf8]{inputenc}
\usepackage[T1]{fontenc}
\usepackage{lmodern,enumitem,multicol,paracol,cancel,float}
\usepackage{mathtools,amsthm,amssymb,icomma,upgreek,xfrac}
\usepackage[hidelinks,breaklinks,pdfusetitle,pdfdisplaydoctitle]{hyperref}
\mathtoolsset{showonlyrefs,mathic}
\title{Pytania na egzamin z procesów stochastycznych}
\date{\today}

\newtheoremstyle{break}
{\topsep}{\topsep}%
{\normalfont}{}%
{\em}{}%
{\newline}{}%
\theoremstyle{plain}
\newtheorem{pytanie}{}
\theoremstyle{break}
\newcounter {odpowiedz}
\newenvironment{odpowiedź}{\refstepcounter {odpowiedz}}{\vspace{0.3em}\hrule}


%% STAŁE
% liczba e
\newcommand*{\e}{\mathrm{e}}


%% OPERATORY:
% `\diff` od „differential”, czyli odpowiednika słowa „różniczka” w języku
% angielskim.
\DeclareMathOperator{\diff}{d\!}
% Część rzeczywista niepisana frakturą, `\Rea` od Real
\DeclareMathOperator{\Rea}{Re}
% Część urojona niepisana frakturą, `\Ima` od Imaginary
\DeclareMathOperator{\Ima}{Im}
\DeclareMathOperator {\expVal} {\mathbb{E}}

\begin{document}
%\maketitle

%%%%%%%%%%%%%%%%%%%%%%%%%%%%%%%%%%%%%%%%
\begin{pytanie}
Co to jest proces stochastyczny?
\end{pytanie}
\begin{odpowiedź}
    Proces stochastyczny to rodzina zmiennych losowych $(X_t)_{t\in I}$
    (gdzie $I$ jest pewnym zbiorem indeksów), określonych na wspólnej
    przestrzeni probabilistycznej. To znaczy \[
    \forall t \in I \quad X_t : \Omega \to \mathbb{R}
    \]
    gdzie $(\Omega, \mathcal{F}, \mathbb{P})$ jest pewną ustaloną przestrzenią
    probabilistyczną.
\end{odpowiedź}
%%%%%%%%%%%%%%%%%%%%%%%%%%%%%%%%%%%%%%%%


%%%%%%%%%%%%%%%%%%%%%%%%%%%%%%%%%%%%%%%%
\begin{pytanie}
Co to jest rozkład wielowymiarowy procesu stochastycznego?
\end{pytanie}
\begin{odpowiedź}
    Dla procesu $(X_t)_{t\in T}$, rozkłady skończenie wymiarowe
    to rozkłady wektorów $(X_{t_1}, X_{t_2}, \ldots, X_{t_n})$
    dla dowolnego $n \in \mathbb{N}$ oraz $t_1, t_2, \ldots t_n \in T$,
    $t_1 < t_2 < \ldots t_n$.
    Czyli rozkłady łączne skończenie wielu zmiennych tworzących proces.
    W~szczególności, jeśli $T$ jest nieskończony, żaden
    rozkład skończenie wymiarowy nie jest rozkładem łącznym
    całego procesu.
\end{odpowiedź}
%%%%%%%%%%%%%%%%%%%%%%%%%%%%%%%%%%%%%%%%


%%%%%%%%%%%%%%%%%%%%%%%%%%%%%%%%%%%%%%%%
\begin{pytanie}
Jak definiuje się własność Markowa procesu stochastycznego?
\end{pytanie}
\begin{odpowiedź}
    Proces stochastyczny $X$ jest procesem Markowa, jeśli $
    \mathbb{P}(X_{n+1}=x_{n+1}\mid X_n=x_n,\ldots ,X_0=x_0)\\
    = \mathbb{P}(X_{n+1}=x_{n+1}\mid X_n=x_n),
    $
    gdzie $n \in \mathbb{N}$ i~$x_0, x_1, \ldots, x_{n+1} \in S$.
    {\bf Nieważne jak gdzieś dotarliśmy, ważne, że tam jesteśmy.}
\end{odpowiedź}
%%%%%%%%%%%%%%%%%%%%%%%%%%%%%%%%%%%%%%%%


%%%%%%%%%%%%%%%%%%%%%%%%%%%%%%%%%%%%%%%%
\begin{pytanie}
Jakie postulaty spełnia macierz przejścia łańcucha Markowa?
\end{pytanie}
\begin{odpowiedź}
    Stochastyczna macierz przejścia: $P = [p(x,y)]_{x, y\in S}$
    \begin{enumerate}
        \item $p(x, y) \geq 0, \quad \forall x, y \in S$
        \item $\sum_{y} p(x, y) = 1, \quad \forall x \in S$
    \end{enumerate}
\end{odpowiedź}
%%%%%%%%%%%%%%%%%%%%%%%%%%%%%%%%%%%%%%%%


%%%%%%%%%%%%%%%%%%%%%%%%%%%%%%%%%%%%%%%%
\begin{pytanie}
Co to znaczy, że macierz jest podwójnie stochastyczna? Proszę podać przykład.
\end{pytanie}
\begin{odpowiedź}
    Macierz podwójnie stochastyczna to taka macierz,
    której wyrazy są nieujemne, oraz
    wiersze i kolumny sumują się do 1.
    Równoważnie macierz stochastyczna,
    której transpozycja jest stochastyczna.
    Na przykład macierz identycznościowa.
\end{odpowiedź}
%%%%%%%%%%%%%%%%%%%%%%%%%%%%%%%%%%%%%%%%


%%%%%%%%%%%%%%%%%%%%%%%%%%%%%%%%%%%%%%%%
\begin{pytanie}
Co to znaczy, że łańcuch Markowa jest nieprzywiedlny?
\end{pytanie}
\begin{odpowiedź}
    Łańcuch Markowa jest nieprzywiedlny, jeżeli wszystkie
    stany komunikują się ze sobą (z każdego stanu
    możemy przejść do innego).
\end{odpowiedź}
%%%%%%%%%%%%%%%%%%%%%%%%%%%%%%%%%%%%%%%%


%%%%%%%%%%%%%%%%%%%%%%%%%%%%%%%%%%%%%%%%
\begin{pytanie}
Jak definiujemy okres łańcucha Markowa?
\end{pytanie}
\begin{odpowiedź}\label{definicja-okresu}
    Niech $D_i = \{n \in \mathbb{N}: p_{ii}^{(n)} > 0 \}$.
    Okres stanu $i$ to $d_i=\operatorname{NWD}(D_i)$.
    Jeśli łańcuch jest nieprzywiedlny, to wszystkie stany mają ten
    sam okres. Oznaczamy wspólną wartość $d = d_i$.
    Wtedy jeśli $d > 1$ to mówimy że łańcuch ma okres $d$.
    W przeciwnym razie łańcuch jest nieokresowy.
\end{odpowiedź}
%%%%%%%%%%%%%%%%%%%%%%%%%%%%%%%%%%%%%%%%


%%%%%%%%%%%%%%%%%%%%%%%%%%%%%%%%%%%%%%%%
\begin{pytanie}
Co to jest rozkład stacjonarny dla łańcucha Markowa?
\end{pytanie}
\begin{odpowiedź}
    Rozkład prawdopodobieństwa $\mu = (\mu (\{x\}) : x\in S)$
    nazywamy rozkładem stacjonarnym macierzy przejścia $\mathbb{P}$,
    gdy $\mu \mathbb{P} = \mu$.
    Rozkład $\mu$ nie zmienia się w czasie.
\end{odpowiedź}
%%%%%%%%%%%%%%%%%%%%%%%%%%%%%%%%%%%%%%%%


%%%%%%%%%%%%%%%%%%%%%%%%%%%%%%%%%%%%%%%%
\begin{pytanie}
Co to znaczy, że łańcuch Markowa jest nieokresowy?
\end{pytanie}
\begin{odpowiedź}
    Łańcuch Markowa jest nieokresowy, jeżeli $d=1$
    (zob. odpowiedź \ref{definicja-okresu}).
\end{odpowiedź}
%%%%%%%%%%%%%%%%%%%%%%%%%%%%%%%%%%%%%%%%


%%%%%%%%%%%%%%%%%%%%%%%%%%%%%%%%%%%%%%%%
\begin{pytanie}
Proszę sformułować twierdzenie o zbieżności łańcuchów Markowa ze skończoną przestrzenią stanów.
\end{pytanie}
\begin{odpowiedź}
    Niech $\mathbb{P} = [p_{ij}]_{i,j=1}^N$
    będzie macierzą stochastyczną (stopnia  N).
    Załóżmy, że dla pewnego $n_0\in \mathbb{N}$
    mamy $p_{ij}^{(n_0)} > 0$ dla wszystkich $i, j=1, \ldots N$.
    Wtedy istnieje wektor $\pi = (\pi_j: j = 1, \ldots N)$
    taki, że $\lim_{n\to \infty}p^{(n)}_{ij} = \pi_j$ dla
    $i = 1, 2, \ldots N$ oraz $\pi$ jest jedynym rozkładem
    stacjonarnym macierzy $\mathbb{P}$.

    Założenia o dodatniości macierzy można równoważnie zastąpić
    założeniem nieprzywiedlności i nieokresowości.
\end{odpowiedź}
%%%%%%%%%%%%%%%%%%%%%%%%%%%%%%%%%%%%%%%%


%%%%%%%%%%%%%%%%%%%%%%%%%%%%%%%%%%%%%%%%
\begin{pytanie}
Jak definiujemy czas trafienia łańcucha Markowa w ustalony punkt?
\end{pytanie}
\begin{odpowiedź}
    Niech $S$ \pauza przeliczalny zbiór stanów i $A\subseteq S$.
    Definiujemy $T_A=\inf\{n\in \mathbb{N} : X_n\in A\}$,
    gdzie $\inf{\emptyset} = \infty$.\\
    Oznaczenie: $T_{\{y\}} = T_y$.
\end{odpowiedź}
%%%%%%%%%%%%%%%%%%%%%%%%%%%%%%%%%%%%%%%%


%%%%%%%%%%%%%%%%%%%%%%%%%%%%%%%%%%%%%%%%
\begin{pytanie}
Co oznacza tranzytywność i rekurencyjność łańcucha Markowa o przeliczalnym zbiorze stanów?
\end{pytanie}
\begin{odpowiedź}
    Niech $S$ będzie przeliczalnym zbiorem stanów.
    Dla $x, y \in S$ definiujemy
    $\rho_{x,y} = \sum_{m = 1}^\infty P_x(T_y = m) = 1 - P_x(T_y = \infty)$.
    Stan $x$ nazywamy powracającym (rekurencyjnym),
    gdy $\rho_{x,x} = 1$,
    a chwilowym (tranzytywnym), jeżeli $\rho_{x,x} < 1$.
\end{odpowiedź}
%%%%%%%%%%%%%%%%%%%%%%%%%%%%%%%%%%%%%%%%


%%%%%%%%%%%%%%%%%%%%%%%%%%%%%%%%%%%%%%%%
\begin{pytanie}
Jak definiujemy generator macierzy przejścia z~czasem ciągłym?
\end{pytanie}
\begin{odpowiedź}
    $(P(t), t\geq 0)$ jest Markowską macierzą stochastyczną
    $N \times N$ z czasem ciągłym wtedy i tylko wtedy,
    gdy istnieje macierz $G= [g_{ij}]$ ($N\times N)$ taka, że
    \begin{gather}
        g_{ij} \geq 0 \,\, i \neq j, \quad
        \sum_{j=1}^N g_{ij} = 0, \quad
        P(t)=\exp(tG) \,\, t\geq 0.
    \end{gather}
    Macierz $G$ jest jedyna i nazywamy ją {\em generatorem} $P(t)$.
    W skrócie: $G$ jest taka, że $P(t) = \exp(tG)$.
\end{odpowiedź}
%%%%%%%%%%%%%%%%%%%%%%%%%%%%%%%%%%%%%%%%


%%%%%%%%%%%%%%%%%%%%%%%%%%%%%%%%%%%%%%%%
\begin{pytanie}
Jakie postulaty spełnia generator macierzy przejścia z czasem ciągłym?
\end{pytanie}
\begin{odpowiedź}
    \begin{gather}
        g_{ij} \geq 0 \,\, i \neq j, \quad
        \sum_{j=1}^N g_{ij} = 0, \quad
        P(t)=\exp(tG) \,\, t\geq 0.
    \end{gather}
\end{odpowiedź}
%%%%%%%%%%%%%%%%%%%%%%%%%%%%%%%%%%%%%%%%


%%%%%%%%%%%%%%%%%%%%%%%%%%%%%%%%%%%%%%%%
\begin{pytanie}
Jakie równanie różniczkowe spełnia macierz stochastyczna z czasem ciągłym?
\end{pytanie}
\begin{odpowiedź}
    \begin{equation}
        P(t)' = GP(t)
    \end{equation}
    gdzie $G$ jest generatorem $P$, czyli $P(t)=\exp(tG)$.
\end{odpowiedź}
%%%%%%%%%%%%%%%%%%%%%%%%%%%%%%%%%%%%%%%%
\pagebreak % rozmieszczenie na stronach


%%%%%%%%%%%%%%%%%%%%%%%%%%%%%%%%%%%%%%%%
\begin{pytanie}
W jaki sposób opisujemy czysty proces urodzin?
\end{pytanie}
\begin{odpowiedź}
    \begin{itemize}
        \item Proces sygnałowy: niemalejący w czasie, càdlàg\\
              (zob. \ref{cadlag}), o~wartościach całkowitych nieujemnych;
        \item Proces o przyrostach niezależnych (zob. \ref{niezależność-przyrostów});
        \item Proces o przyrostach jednorodnych w czasie:\\
              $X(s+t)-X(s)\overset d = X(t)-X(0)$.
    \end{itemize}
\end{odpowiedź}
%%%%%%%%%%%%%%%%%%%%%%%%%%%%%%%%%%%%%%%%


%%%%%%%%%%%%%%%%%%%%%%%%%%%%%%%%%%%%%%%%
\begin{pytanie}
Co oznacza eksplozja demograficzna dla czystego procesu urodzin?
\end{pytanie}
\begin{odpowiedź}
    Eksplozja demograficzna czystego procesu urodzin to sytuacja,
    gdzie w skończonym czasie przychodzi nieskończenie wiele
    sygnałów, czyli $\mathbb{P}(X_t = \infty) > 0$ dla pewnego $t < \infty$.
\end{odpowiedź}
%%%%%%%%%%%%%%%%%%%%%%%%%%%%%%%%%%%%%%%%


%%%%%%%%%%%%%%%%%%%%%%%%%%%%%%%%%%%%%%%%
\begin{pytanie}
Proszę sformułować twierdzenie o braku eksplozji dla czystego procesu urodzin.
\end{pytanie}
\begin{odpowiedź}
    Niech $X(t)$ będzie czystym procesem urodzin
    o intensywnościach $\lambda_n \geq 0$ i~stanie początkowym 0,
    tzn. $\mathbb{P}(X(0)=0)=1$. Zachodzi \[
        \left(\forall t \geq 0 \sum_{n = 0}^\infty
        \mathbb{P}(X(t)=n)=1\right) \quad \iff \quad
        \sum_{n = 0}^\infty \frac 1 {\lambda_n} = \infty
    \]
\end{odpowiedź}
%%%%%%%%%%%%%%%%%%%%%%%%%%%%%%%%%%%%%%%%


%%%%%%%%%%%%%%%%%%%%%%%%%%%%%%%%%%%%%%%%
\begin{pytanie}
Proszę podać postulaty procesu Poissona.
\end{pytanie}
\begin{odpowiedź}
    \begin{enumerate}
        \item $X(0) = 0$ p.n.p.
        \item Niezależność przyrostów (zob. \ref{niezależność-przyrostów})
        \item $X(s+t)-X(s)$ ma rozkład Poissona o średniej $\lambda t$
        \item Trajektorie procesu $X$ są prawostronnie ciągłe
            na $[0, \infty)$, mają lewostronne granice na $(0, \infty )$
            o skokach wysokości $1$.
    \end{enumerate}
\end{odpowiedź}
%%%%%%%%%%%%%%%%%%%%%%%%%%%%%%%%%%%%%%%%


%%%%%%%%%%%%%%%%%%%%%%%%%%%%%%%%%%%%%%%%
\begin{pytanie}
Co oznacza niezależność przyrostów procesu stochastycznego?
\end{pytanie}
\begin{odpowiedź} \label{niezależność-przyrostów}
    Niezależność przyrostów:
    $X(t_i)-X(s_i)$ dla $i=1,2,\ldots,n$
    są niezależne dla rozłącznych przedziałów $(s_i,t_i)$.
\end{odpowiedź}
%%%%%%%%%%%%%%%%%%%%%%%%%%%%%%%%%%%%%%%%


%%%%%%%%%%%%%%%%%%%%%%%%%%%%%%%%%%%%%%%%
\begin{pytanie}
Jak definiujemy funkcję charakterystyczną, funkcję tworzącą momenty i transformatę Laplace'a zmiennej losowej?
\end{pytanie}
\begin{odpowiedź}
    Transformata Laplace'a: Jeżeli $X$ jest zmienną losową,
    to $\phi(t) = \mathbb{E}(\e^{-tX}), \ t \geq 0$
    nazywany jest transformatą Laplace'a zmiennej losowej X.
    Funkcja tworząca momenty: $\operatorname{MGF}_X(t)
    = \mathbb{E} (\e^{tX}), \ t \in (-\varepsilon, \varepsilon)$. \\
    Funkcja charakterystyczna: $\varphi_X(t)
    = \mathbb{E} (\e^{itX}), \ t \in \mathbb{R}$.
\end{odpowiedź}
%%%%%%%%%%%%%%%%%%%%%%%%%%%%%%%%%%%%%%%%


%%%%%%%%%%%%%%%%%%%%%%%%%%%%%%%%%%%%%%%%
\begin{pytanie}
Czy zna Pani/Pan jakąś konstrukcję procesu Poissona? Proszę o niej opowiedzieć.
\end{pytanie}
\begin{odpowiedź}
    Niech $T_1, T_2, \ldots$ będą iid. $\operatorname{Exp}(\lambda)$.
    Niech $S_n = T_1 + T_2 + \ldots T_n$.
    Wtedy $N(t) =\sup\{n \geq 0: S_n \leq t\}$ jest procesem Poissona
    z parametrem $\lambda$.
    Omawialiśmy tę konstrukcję zanim zdefiniowaliśmy proces aksjomatycznie.
    Zachodzi $N(t) = n \iff S_n \leq t < S_{n+1}$ oraz $N(t) \geq n \iff S_n < t$.
\end{odpowiedź}
%%%%%%%%%%%%%%%%%%%%%%%%%%%%%%%%%%%%%%%%
\pagebreak % rozmieszczenie na stronach


%%%%%%%%%%%%%%%%%%%%%%%%%%%%%%%%%%%%%%%%
\begin{pytanie}
Proszę podać rozkłady jednowymiarowe procesu Poissona.
\end{pytanie}
\begin{odpowiedź}
    Niech $(X_t)_{t\geq 0}$ będzie procesem Poissona.
    Rozkłady jednowymiarowe, to rozkłady wektora jednoelementowego
    $X_t$ (wektora z tylko jedną współrzędną) dla różnych wyborów $t$.
    Czyli jest to rozkład Poissona z parametrem $\lambda t$.
\end{odpowiedź}
%%%%%%%%%%%%%%%%%%%%%%%%%%%%%%%%%%%%%%%%


%%%%%%%%%%%%%%%%%%%%%%%%%%%%%%%%%%%%%%%%
\begin{pytanie}
Proszę podać funkcję charakterystyczną rozkładu jednowymiarowego procesu Poissona.
\end{pytanie}
\begin{odpowiedź}
    Funkcja charakterystyczna: \[
    \varphi(\xi) = \mathbb{E}\e^{i\xi N(t)} = \exp[\lambda t (\e^{i\xi} - 1)]
    \]
\end{odpowiedź}
%%%%%%%%%%%%%%%%%%%%%%%%%%%%%%%%%%%%%%%%


%%%%%%%%%%%%%%%%%%%%%%%%%%%%%%%%%%%%%%%%
\begin{pytanie}
Jakie postulaty spełnia prawdopodobieństwo przejścia?
\end{pytanie}
\begin{odpowiedź}
\begin{enumerate}
    \item $A \mapsto P_{s,t} (x,A)$ są miarami probabilistycznymi;
    \item $x \mapsto P_{s,t}(x,A)$ są funkcjami mierzalnymi;
    \item równania Chapmana-Kołmogorowa: dla $s \leq t \leq u$, \[
    \int_S p_{s,t}(x, \diff y) p_{t, u}(y, A) = p_{s, u}(x, A);
    \]
    \item Dla $s>t$, $p_{s, t} = 0$, dla $s = t$, $p_{s,s}(x, \cdot) = \delta_x$.
\end{enumerate}
\end{odpowiedź}
%%%%%%%%%%%%%%%%%%%%%%%%%%%%%%%%%%%%%%%%


%%%%%%%%%%%%%%%%%%%%%%%%%%%%%%%%%%%%%%%%
\begin{pytanie}
Co to znaczy, że prawdopodobieństwo przejścia jest jednorodne w czasie?
\end{pytanie}
\begin{odpowiedź}
    Prawdopodobieństwo przejścia jest jednorodne w~czasie,
    gdy $T=\mathbb{R}$ oraz
    \[p_{s, t} (x,A) = p_{0, t-s} (x,A).\]
\end{odpowiedź}
%%%%%%%%%%%%%%%%%%%%%%%%%%%%%%%%%%%%%%%%


%%%%%%%%%%%%%%%%%%%%%%%%%%%%%%%%%%%%%%%%
\begin{pytanie}
Co to znaczy, że prawdopodobieństwo przejścia jest jednorodne w przestrzeni?
\end{pytanie}
\begin{odpowiedź}
    Prawdopodobieństwo przejścia jest jednorodne w~przestrzeni,
    jeżeli ,,przestrzeń'' jest przestrzenią wektorową oraz \[
        (\forall x, y, A) \qquad p_{s, t}(x, A) = p_{s, t}(x + y, A + y).
    \]
\end{odpowiedź}
%%%%%%%%%%%%%%%%%%%%%%%%%%%%%%%%%%%%%%%%


%%%%%%%%%%%%%%%%%%%%%%%%%%%%%%%%%%%%%%%%
\begin{pytanie}
Proszę podać prawdopodobieństwo przejścia procesu Poissona.
\end{pytanie}
\begin{odpowiedź}
    P.p. dla singletonu $\{n + k\}$ :
    \begin{equation}
        p_t(n, \{n + k\}) = \mathbb{P}(X(t) = n + k \mid X(0) = n)
        = \frac {(t\lambda)^k}{k!} \e^{-t\lambda}.
    \end{equation}
    Dla dowolnego zbioru $A \subseteq \mathbb{N}_0$, mamy\\
    $p_t(n, A) = \sum_{k \in A, k \geq n} p_t(n, \{k\})$.
\end{odpowiedź}
%%%%%%%%%%%%%%%%%%%%%%%%%%%%%%%%%%%%%%%%


%%%%%%%%%%%%%%%%%%%%%%%%%%%%%%%%%%%%%%%%
\begin{pytanie}
Co to jest trajektoria procesu stochastycznego?
\end{pytanie}
\begin{odpowiedź}
    Trajektoria: $[0, \infty) \ni t \mapsto N(t) = N(t, \omega)$.
    Czyli dla ustalonego $\omega \in \Omega$, jest to funkcja
    $t \mapsto N(t)(\omega)$.
\end{odpowiedź}
%%%%%%%%%%%%%%%%%%%%%%%%%%%%%%%%%%%%%%%%


%%%%%%%%%%%%%%%%%%%%%%%%%%%%%%%%%%%%%%%%
\begin{pytanie}
Co to znaczy, że trajektorie procesu są càdlàg?
\end{pytanie}
\begin{odpowiedź}\label{cadlag}
    càdlàg – prawostronnie ciągłe z lewostronnymi granicami
    (continue a droite, limit a gauche)
\end{odpowiedź}
%%%%%%%%%%%%%%%%%%%%%%%%%%%%%%%%%%%%%%%%


%%%%%%%%%%%%%%%%%%%%%%%%%%%%%%%%%%%%%%%%
\begin{pytanie}
Co oznacza ciągłość stochastyczna procesu? Czy proces Poissona jest stochastycznie ciągły?
\end{pytanie}
\begin{odpowiedź}
    Proces stochastyczny $(X(t))_{t \in T}$ jest stochastycznie
    ciągły, jeśli $t_n \to t$ implikuje $X(t_n) \overset {\mathbb{P}}\to X(t)$
    (zbieżność według prawdopodobieństwa), czyli
    $\mathbb{P}(|X(t_n) - X(t)| > \varepsilon) \to 0$ gdy $t_n \to t$ dla
    dowolnego $\varepsilon > 0$. Proces Poissona jest stochastycznie
    ciągły (zadanie 6 z listy 7).
\end{odpowiedź}
%%%%%%%%%%%%%%%%%%%%%%%%%%%%%%%%%%%%%%%%


%%%%%%%%%%%%%%%%%%%%%%%%%%%%%%%%%%%%%%%%
\begin{pytanie}
Proszę podać postulaty procesu Wienera.
\end{pytanie}
\begin{odpowiedź}
\begin{enumerate}
    \item $W(0) = 0$ p.n.p.
    \item $W$ ma przyrosty niezależne (zob. \ref{niezależność-przyrostów})
    \item $W(t+s) - W(s) \overset D = N(0, t)$
    \item Trajektorie $t \mapsto W(t)$ są ciągłe p.n.p.
\end{enumerate}
\end{odpowiedź}
%%%%%%%%%%%%%%%%%%%%%%%%%%%%%%%%%%%%%%%%


%%%%%%%%%%%%%%%%%%%%%%%%%%%%%%%%%%%%%%%%
\begin{pytanie}
Proszę podać rozkłady jednowymiarowe procesu Wienera.
\end{pytanie}
\begin{odpowiedź}
    Rozkład jednowymiarowy to rozkład wektora z tylko jedną współrzędną,
    czyli rozkład $W(t)$ dla ustalonego (dowolnego) $t \geq 0$.
    Z postulatów procesu Wienera wynika, że musi być to $N(0, t)$.
\end{odpowiedź}
%%%%%%%%%%%%%%%%%%%%%%%%%%%%%%%%%%%%%%%%


%%%%%%%%%%%%%%%%%%%%%%%%%%%%%%%%%%%%%%%%
\begin{pytanie}
Proszę podać prawdopodobieństwo przejścia procesu Wienera.
\end{pytanie}
\begin{odpowiedź}
\[
    P_t(x, A) = \int_{A - x}\frac 1 {\sqrt{2\pi t}} \e^{-y^2/2t} \diff y
\] 
\end{odpowiedź}
%%%%%%%%%%%%%%%%%%%%%%%%%%%%%%%%%%%%%%%%


%%%%%%%%%%%%%%%%%%%%%%%%%%%%%%%%%%%%%%%%
\begin{pytanie}
Jak definiujemy warunek Höldera ciągłości funkcji? Czy trajektorie Procesu Wienera spełniają ten warunek?
\end{pytanie}
\begin{odpowiedź}
    Niech $p \in (0, 1/2)$. Wtedy istnieje zmienna losowa
    $C: \Omega \to [0, \infty)$ taka, że
    $|W_t - W_s| \leq  C|t - s|^p$, dla $0 \leq s, t, \leq 1$.
    Proces Wienera spełnia ten warunek. Dla funkcji deterministycznych,
    $C$ musi być stałą. Istnienie $C$ jest definicją ciągłości Höldera.
\end{odpowiedź}
%%%%%%%%%%%%%%%%%%%%%%%%%%%%%%%%%%%%%%%%


%%%%%%%%%%%%%%%%%%%%%%%%%%%%%%%%%%%%%%%%
\begin{pytanie}
Jak definiujemy odwrotkę procesu Wienera?
\end{pytanie}
\begin{odpowiedź}
    Jeśli $W_t$ jest procesem Wienera, to \[
    X_t = \begin{cases}
        tW\left(\frac 1 t\right) \quad &:t>0 \\
        0 \quad &: t = 0
        \end{cases}
    \] też jest procesem Wienera.
\end{odpowiedź}
%%%%%%%%%%%%%%%%%%%%%%%%%%%%%%%%%%%%%%%%


%%%%%%%%%%%%%%%%%%%%%%%%%%%%%%%%%%%%%%%%
\begin{pytanie}
Proszę sformułować prawo iterowanego logarytmu.
\end{pytanie}
\begin{odpowiedź}
    Niech $(W_t, t \geq 0)$ będzie procesem Wienera. Wtedy \[
    \limsup_{t \to 0^+} \frac {W_t} {\sqrt{2t \ln \ln \frac 1 t}}
    = 1 \quad \text{p.n.p.}
    \]
\end{odpowiedź}
%%%%%%%%%%%%%%%%%%%%%%%%%%%%%%%%%%%%%%%%


%%%%%%%%%%%%%%%%%%%%%%%%%%%%%%%%%%%%%%%%
\begin{pytanie}
Co to jest filtracja?
\end{pytanie}
\begin{odpowiedź}
    W rachunku prawdopodobieństwa mówimy, że rodzina sigma\dywiz ciał \(
    \left\{ \mathcal F_i \right\}_{i \in I} \), gdzie \( I \) jest w pełni
    uporządkowanym zbiorem indeksów, jest filtracją, jeżeli
    \begin {equation*}
        \left(\forall i, j \in I \right) \left[ i \leqslant j
            \implies \mathcal F_i \subseteq \mathcal F_j\right].
    \end {equation*}
\end{odpowiedź}
%%%%%%%%%%%%%%%%%%%%%%%%%%%%%%%%%%%%%%%%


%%%%%%%%%%%%%%%%%%%%%%%%%%%%%%%%%%%%%%%%
\begin{pytanie}
Proszę sformułować prawo 0-1 Blumenthala dla procesu Wienera.
\end{pytanie}
\begin{odpowiedź}
    Niech $(W_t, t\geq 0)$ będzie procesem Wienera. Niech
    $\mathcal{B}_u = \sigma(W_t, t\leq u)$ dla $u \geq 0$.
    Niech $\mathcal{B}_{u^+} = \bigcap_{t > u} \mathcal{B}_t$.
    Prawo 0-1 Blumenthala mówi, że sigma\dywiz ciało $ \mathcal{B}_{0^+}$
    jest zdegenerowane, czyli wszystkie zdarzenia $A \in \mathcal{B}_{0^+}$
    mają prawdopodobieństwo $0$ lub $1$.
\end{odpowiedź}
%%%%%%%%%%%%%%%%%%%%%%%%%%%%%%%%%%%%%%%%


%%%%%%%%%%%%%%%%%%%%%%%%%%%%%%%%%%%%%%%%
\begin{pytanie}
Co to jest martyngał?
\end{pytanie}
\begin{odpowiedź} \label {def-martyngału}
    Rozważmy filtrowaną przestrzeń probabilistyczną z sigma\dywiz ciałem \(
    \mathcal F_t \) należącym do filtracji i zdefiniowany na niej proces
    stochastyczny \( \left\{ X_t \right\}_{t \in T} \). Mówimy, że \( X_t
    \) jest \textit {martyngałem}, jeżeli dla każdego \( t \in T \), \( X_t
    \) jest \( \mathcal F_t \) mierzalne, ma skończoną wartość oczekiwaną
    \( \expVal |X_t| < \infty \) oraz
    \begin {equation*}
        \left(\forall s, t \in T, s \leq t \right) \expVal (X_t | \mathcal F_s) = X_s.
        \quad \text{p.n.p.}.
    \end {equation*}
\end{odpowiedź}
%%%%%%%%%%%%%%%%%%%%%%%%%%%%%%%%%%%%%%%%


%%%%%%%%%%%%%%%%%%%%%%%%%%%%%%%%%%%%%%%%
\begin{pytanie}
Proszę podać przykład martyngału.
\end{pytanie}
\begin{odpowiedź} \label {przykład-martyngału}
    Błądzenie losowe z naturalną filtracją. Ustalmy ciąg niezależnych
    zmiennych losowych \( \left( X_n \right)_{n \in \mathbf N}, \expVal X_n
    = x < \infty \) oraz sum częściowych \( S_n \coloneqq \sum_{i \leq n}
    X_i \), a także sigma\dywiz ciało \( \mathcal F_t \) z filtracji. Wtedy
    dla \( m < n \) mamy
    \begin {equation*}
    \expVal (S_n | \mathcal F_m) = \expVal \left(\left. S_m + \sum_{i=m +
        1}^n X_i \right| \mathcal F_m \right) = S_m + (n - m)x.
    \end {equation*}
    Dla \( x = 0 \), \( S_m \) jest martyngałem.
\end{odpowiedź}
%%%%%%%%%%%%%%%%%%%%%%%%%%%%%%%%%%%%%%%%


%%%%%%%%%%%%%%%%%%%%%%%%%%%%%%%%%%%%%%%%
\begin{pytanie}
Co to jest nadmartyngał?
\end{pytanie}
\begin{odpowiedź} \label {def-nadmartyngału}
    Przyjmijmy założenia takie, jak w odpowiedzi \ref
    {def-martyngału}. Zmianie ulega jedynie relacja pomiędzy wartością
    oczekiwaną, a zmienną losową. To znaczy mamy
    \begin {equation*}
        \left(\forall s, t \in T, s \leq t \right) \expVal (X_t | \mathcal
        F_s) \leqslant X_s \quad \text{p.n.p.}
    \end {equation*}
\end{odpowiedź}
%%%%%%%%%%%%%%%%%%%%%%%%%%%%%%%%%%%%%%%%


%%%%%%%%%%%%%%%%%%%%%%%%%%%%%%%%%%%%%%%%
\begin{pytanie}
Proszę podać przykład nadmartyngału, który nie jest martyngałem.
\end{pytanie}
\begin{odpowiedź}
    Trywialny przykład: rozważmy trywialną filtrację
    $\mathcal{F}_t = \{\emptyset, \Omega\}$ oraz zmienne losowe
    $X_t = -t$ (deterministyczne). Wtedy  \[
    \mathbb{E}(X_t\mid \mathcal{F}_s) = \mathbb{E}X_t = -t \leq -s
    \] 
    o ile tylko $t \geq s$. Nie jest to martyngał, bo dla $t > s$
    nierówność jest ostra.

    Kolejny przykład jest w odpowiedzi \ref {przykład-martyngału} dla \( x
    < 0 \).
\end{odpowiedź}
%%%%%%%%%%%%%%%%%%%%%%%%%%%%%%%%%%%%%%%%


%%%%%%%%%%%%%%%%%%%%%%%%%%%%%%%%%%%%%%%%
\begin{pytanie}
Co to jest podmartyngał?
\end{pytanie}
\begin{odpowiedź}
    Przyjmijmy założenia takie, jak w odpowiedzi \ref
    {def-martyngału}. Zmianie ulega jedynie relacja pomiędzy wartością
    oczekiwaną, a zmienną losową. To znaczy mamy
    \begin {equation*}
        \left(\forall s, t \in T, s \leq t \right) \expVal (X_t | \mathcal
        F_s) \geqslant X_s\quad \text{p.n.p.}
    \end {equation*}
\end{odpowiedź}
%%%%%%%%%%%%%%%%%%%%%%%%%%%%%%%%%%%%%%%%
\pagebreak % rozmieszczenie na stronach


%%%%%%%%%%%%%%%%%%%%%%%%%%%%%%%%%%%%%%%%
\begin{pytanie}
Proszę sformułować nierówność Jensena dla warunkowych wartości oczekiwanych.
\end{pytanie}
\begin{odpowiedź}
    Niech $g$ będzie funkcją wypukłą. Niech $X$ oraz $g(X)$ będą
    zmiennymi losowymi całkowalnymi, czyli takimi, że
    $\mathbb{E}|X|, \mathbb{E}|g(X)| < \infty$.
    Wtedy zachodzi nierówność Jensena: \[
    g(\mathbb{E}[X\mid \mathcal{B}]) \leq \mathbb{E}[g(X) \mid \mathcal{B}]
    \quad \text{p.n.p.}
    \]
    dla dowolnego sigma\dywiz ciała $\mathcal{B}$.
\end{odpowiedź}
%%%%%%%%%%%%%%%%%%%%%%%%%%%%%%%%%%%%%%%%


%%%%%%%%%%%%%%%%%%%%%%%%%%%%%%%%%%%%%%%%
\begin{pytanie}
Jak definiujemy czas stopu?
\end{pytanie}
\begin{odpowiedź}
    Zmienna losowa $\tau : \Omega \to T \cup \{\infty\}$ jest
    {\emph czasem stopu} względem filtracji
    $(\mathcal{F}_t, t \in T)$ jeśli \[
    \forall t \in T \quad \{\tau \leq t\} \in \mathcal{F}_t.
    \]
\end{odpowiedź}
%%%%%%%%%%%%%%%%%%%%%%%%%%%%%%%%%%%%%%%%


%%%%%%%%%%%%%%%%%%%%%%%%%%%%%%%%%%%%%%%%
\begin{pytanie}
Proszę sformułować twierdzenie Dooba o stopowaniu.
\end{pytanie}
\begin{odpowiedź}
    Niech $(Z_n)$ będzie martyngałem z czasem dyskretnym.
    Niech $N$ będzie skończonym momentem stopu.
    Jeśli zachodzi choć jeden z warunków
    \begin{enumerate}
        \item $\bar Z_n = Z_{n\land N}$ są wspólnie ograniczone
        \item $N$ jest ograniczone
        \item Istnieje $M$ takie, że
            $\mathbb{E}[|Z_n - Z_{n-1}|\mid \mathcal{F}_n] \leq M$
            p.n.p. oraz $\mathbb{E}N < \infty$
    \end{enumerate}
    wtedy $\mathbb{E}Z_N = \mathbb{E}Z_1$.
\end{odpowiedź}
%%%%%%%%%%%%%%%%%%%%%%%%%%%%%%%%%%%%%%%%


%%%%%%%%%%%%%%%%%%%%%%%%%%%%%%%%%%%%%%%%
\begin{pytanie}
Proszę sformułować tożsamość Walda.
\end{pytanie}
\begin{odpowiedź}
    Niech $X_i$ będą iid., oraz $\mathbb{E}|X_1| < \infty$.
    Niech $N$ będzie czasem stopu takim, że $\mathbb{E}N < \infty$.
    Wtedy \[
    \mathbb{E}\left( \sum_{i = i}^N X_i \right)=
    \left(\mathbb{E}X_1 \right)\cdot \left(\mathbb{E}N \right)
    \]
\end{odpowiedź}
%%%%%%%%%%%%%%%%%%%%%%%%%%%%%%%%%%%%%%%%


\end{document}
